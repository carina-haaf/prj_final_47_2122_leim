%%________________________________________________________________________
%% LEIM | PROJETO
%% 2022 / 2013 / 2012
%% Modelo para relat�rio
%% v04: altera��o ADEETC para DEETC; outros ajustes
%% v03: corre��o de gralhas
%% v02: inclui anexo sobre utiliza��o sistema controlo de vers�es
%% v01: original
%% PTS / MAR.2022 / MAI.2013 / 23.MAI.2012 (constru�do)
%%________________________________________________________________________




%%________________________________________________________________________
\myPrefaceChapter{Resumo}
%%________________________________________________________________________

O desempenho de um atleta, numa determinada modalidade desportiva, melhora quando este � acompanhado de uma perspetiva externa no decurso da sua atividade desportiva. Neste sentido, o atleta pode ser monitorizado por um treinador para atingir melhores resultados. Como complemento, � poss�vel registar em v�deo e analisar posteriormente o desempenho do atleta.

Neste sentido, � vantajoso o desenvolvimento de uma ferramenta capaz de realizar essa an�lise externa. A ferramenta desenvolvida neste projeto, permite extrair e reconhecer eventos relevantes (i.e. per�odos em que ocorra uma maior troca de bolas) com base no v�deo da atividade desportiva do atleta (em treinos ou competi��es de padel /t�nis). O processamento � realizado com base no �udio extra�do do v�deo, e tem como base a extra��o de caracter�sticas ou padr�es identificativos dos eventos, com o aux�lio de t�cnicas de aprendizagem autom�tica, baseadas em s�ries temporais. No final, s�o realizadas estat�sticas, que permitem obter um resumo detalhado do que foi registado no v�deo, dando uma perspetiva mais abrangente e objetiva do desempenho do atleta. A ferramenta identifica corretamente cerca de 85\% dos eventos em an�lise, sendo necess�rios alguns ajustes para melhorar o processo de reconhecimento.

\bigskip
Os ensaios s�o realizados no Laborat�rio de �udio e Ac�stica do ISEL, LAA.

\bigskip
\bigskip
\bigskip
\textbf{Palavras-chave:} algoritmos de software de dete��o do som, algoritmos de software de intelig�ncia artificial.






%%________________________________________________________________________
\myPrefaceChapter{Abstract}
%%________________________________________________________________________

In a given sport, the athlete's performance improves when he is supervised from an external perspective. In this scenario, the athlete can be supervised by a coach in order to achieve better results, or as a complement, it is possible to video record and analyze his performance afterwards.

Thus, it is advantageous to develop a tool capable of performing this external analysis. The tool developed in this project, allows extracting and recognizing relevant events (i.e., periods when a greater exchange of balls occurs) based on the video of the athlete's sport activity (in the padel/tennis sport activity). The processing is done based on the audio extracted from the video, and is based on the extraction of patterns identifying the events, with the help of machine learning techniques, based on time series. At the end, statistics are performed, allowing a detailed summary of what was recorded in the video, giving a more comprehensive and objective perspective of the athlete's performance. The tool correctly identifies about 85\% of the events under analysis, with some adjustments needed to improve the recognition process.

\bigskip
The tests are carried out at ISEL's Audio and Acoustics Laboratory, LAA.

\bigskip
\bigskip
\bigskip
\textbf{Keywords:} sound detection software algorithms, artificial intelligence software algorithms.
